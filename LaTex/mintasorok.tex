% Táblák mintasorokkal

\subsection{SZEREP tábla}
A \tabla{szerep} táblában a rendszer felhasználóihoz rendelet felhasználói 
szerepek értékkészlete jelenik meg.\\

\tabla{SZEREP}(\pk{szerep}, aktív szerep)

\begin{table}[ht!]
\centering
{\footnotesize \begin{tabular}[t]{|l|l|}
\hline
 \textbf{szerep}&aktív\\\hline
 \textbf{admin}&igen\\
 \textbf{kalibráló}&igen\\
 \textbf{\dots}&\dots\\
\end{tabular}}
\caption{SZEREP tábla} \label{tabSZEREP}
\end{table}


\subsection{MINŐSÍTÉS tábla}
A tábla tartalmazza a műszerek minősítéséhez használható értékeket.\\

\tabla{MINŐSÍTÉS}(\pk{minősítés}, aktív)
\begin{table}[ht!]
\centering
\begin{footnotesize}
\begin{tabular}[t]{|l|l|}
\hline
 \textbf{minősítés}&aktív\\\hline
 \textbf{megfelelt}&igen\\
 \textbf{tájékoztató mérésre}&igen\\
 \textbf{selejt}&igen\\
 \hline
\end{tabular}
\end{footnotesize}
\caption{MINŐSÍTÉS tábla} \label{tabMINOSITES}
\end{table}


\subsection{KÖLTSÉGHELY tábla}
A tábla azokat a költséghelyeket tartalmazza, amelyekhez dolgozók, vagy 
műszerek rendelhetők. Egyes költséghelyek idővel megszűnnek, újak jönnek
létre.\\

\tabla{KÖLTSÉGHELY}(\pk{költséghely}, költséghely neve, sorrend, aktív)


\begin{table}[ht!]
\centering
	\begin{footnotesize}
\begin{tabular}[t]{|l|l|l|l|}
\hline
 \textbf{költséghely}&költséghely neve&sorrend&aktív\\ \hline
 \textbf{13-421-0}&Járműlakatos művezetőség&1&igen\\
 \textbf{13-443-0}&Műszerész művezetőség&1&igen\\
 \textbf{11-210-0}&Humánügyi és bérelsz. Oszt.&9&nem\\
 \textbf{\dots}&\dots&\dots&\dots \\
\end{tabular}
\end{footnotesize}
\caption{KÖLTSÉGHELY tábla} \label{tabKOLTSEGHELY}
\end{table}


\subsection{FELHASZNÁLÓ tábla}
A táblában a rendszer felhasználóit tartjuk nyilván. A rendszerhez csak 
jogosult felhasználó férhet hozzá. A felhasználókról a minimális adatokat 
tartunk nyilván\\

\tabla{felhasználó}(\pk{felhasználónév}, vezetéknév, keresztnév, 
harmadik név, titulus, jelszó, \fk{szerep}, aktív felhasználó, kezdődátum, végdátum)


\begin{table}[ht!]
 \centering
 \begin{footnotesize}
 \begin{tabular}[t]{|l|l|l|l|l|l|l|l|}
  \hline
\textbf{felhasználónév}&vezetéknév&\dots&\fk{szerep}
&aktív&kezdődátum&végdátum\\ \hline
  \textbf{nagye}&Nagy&\dots&\fk{admin}&igen&2020.06.01&null \\
  \textbf{kissg}&Kiss&\dots&\fk{kalibráló}&igen&2020.06.01&2021.04.24\\
  \textbf{tothb}&Tóth&\dots&\fk{lekérdező}&nem&2020.06.01&2020.09.06\\
 \end{tabular}
\end{footnotesize}
\caption{FELHASZNÁLÓ tábla}\label{tabFEHASZNALO}
\end{table}


\subsection{DOLGOZÓ tábla}
A tábla tartalmazza azokat a dolgozókat, akik mérőeszközökkel 
rendelkeznek, vagy rendelkeztek. A dolgozókról a minimális, a rendszer 
működése szempontjából fontos adatokat tartjuk csak nyilván.\\

\tabla{dolgozó}(\pk{törzsszám}, vezetéknév, keresztnév, harmadik név, 
\fk{költséghely}, aktív)


\begin{table}[ht!]
 \centering
 \begin{footnotesize}
 \begin{tabular}[t]{|l|l|l|l|l|}
  \hline
\textbf{törzsszám}&vezetéknév&\dots&\fk{költséghely}&aktív\\ 
\hline
  \textbf{93456}&Nagy&\dots&\fk{13-421-0}&igen \\
  \textbf{92312}&Tóth&\dots&\fk{13-443-0}&nem \\
  \textbf{95678}&Kiss&\dots&\fk{13-422-0}&igen\\
 \end{tabular}
\end{footnotesize}
\caption{DOLGOZÓ tábla}\label{tabDOLGOZO}
\end{table}


\subsection{CIKKTÖRZS tábla}
A táblában a SAP rendszer adatain túl olyan adatok is tárolása kerülnek, 
amelyekre a SAP rendszer nem ad lehetőséget, viszont a műszerek
szempontjából lényegesek.\\

\tabla{cikktörzs}(\pk{cikkszám}, megnevezés, típus, tartomány alsó határ, tartomány felső határ, osztás, pontosság, tolerancia+, tolerancia-, kalibrálási gyakoriság, aktív)

\begin{table}[ht!]
	\centering
	\begin{footnotesize}
	\begin{tabular}[t]{|l|l|l|r|r|l|}
		\hline
		\textbf{cikkszám}&megnevezés&\dots&pontosság&kalibr.gyak&aktív\\ 
		\hline
		\textbf{MEV0000001-1}&Mitutoyo tolómérő&\dots&0,01&365&igen \\
		\textbf{MEV0000036-1}&Maxwell multiméter&\dots&\,&365&igen \\
		\textbf{MEV0000117-1}&Sauter mérőcella&\dots&0,1&180&igen \\
	\end{tabular}
\end{footnotesize}
	\caption{CIKKTÖRZS tábla}\label{tabCIKKTORZS}
\end{table}

\subsection{MŰSZER tábla}
Míg a \tabla{CIKKTÖRZS} tábla az azonos műszerek fő jellemzőit tartalmazza, addig a \tabla{műszer} tábla az egyedileg is azonosított eszközök tulajdonságainak rögzítésére hivatott.
Az azonos műszerek között is lehetnek eltérő paraméterekkel rendelkező eszközök, ahogy a kalibrálás gyakorisága függhet a felhasználás intenzitásától, de egy-egy műszer lehet nem kalibrált státuszú, de akár selejt is.
A \tabla{MŰSZER} tábla a \tabla{cikktörzs} tábla specializációja.\\

\tabla{műszer}(\pk{egyedi szám}, \fk{cikkszám}, \dots, )

\begin{table}[ht!]
	\centering
	\begin{footnotesize}
	\begin{tabular}[t]{|l|l|c|}
		\hline
		\textbf{egyedi szám}&\textit{cikkszám}&\dots \\ \hline
		\textbf{110123}&\textit{MEV0000000-1}&\dots\\
		\textbf{134001}&\textit{MEV0000036-1}&\dots\\
		\textbf{\dots}&\textit{\dots}&\dots\\
	\end{tabular}
\end{footnotesize}
	\caption{MŰSZER tábla}\label{tabMUSZER}
\end{table}


\subsection{NYILVÁNTARTÁS tábla}
A táblázat a műszerek dolgozók és raktár közötti mozgásait tartalmazza. A dolgozó a raktárból egy adott napon felveszi a műszert, majd egy másik napon, de akár még ugyanazon a napon visszaviszi és leadja azt. A táblázat segítségével egy adott pillanatban meg tudjuk mondani, melyik műszer éppen hol van, de azt is, de azt is kinél volt egy adott napon. \\

\tabla{nyilvántartás}(\pk{sorszám}, \fk{törzsszám}, \fk{egyedi szám}, mettől, meddig, \dots)\\

\begin{table}[ht!]
	\centering
	\begin{footnotesize}
	\begin{tabular}[t]{|r|c|c|c|c|c|}
		\hline
		\textbf{sorszám}&\textit{törzsszám}&\textit{egyedi szám}&mettől&meddig&\dots \\ \hline
		\textbf{1}&\textit{93456}&\textit{110123}&2020-06-07 07:12:35&null&\dots\\
		\textbf{2}&\textit{93456}&\textit{134001}&2020-06-07 07:12:35&2020-08-01 13:50:47&\dots\\
		\textbf{3}&\textit{95221}&\textit{134001}&2020-08-15 10:41:05&null&\dots\\
		\textbf{\dots}&\textit{\dots}&\textit{\dots}&\dots&\dots&\dots\\
	\end{tabular}
\end{footnotesize}
	\caption{NYILVÁNTARTÁS tábla}\label{tabNYILVANTARTAS}
\end{table}


\begin{minipage}[t]{\linewidth}
\textsf{
		{\footnotesize \textbf{Az idő kezelésének problémája.}
	Az idő mint folytonos adat az adatbázis szempontjából ,,szerencsétlen'' ismertet. 
	Meg\-fi\-gyel\-he\-tő az a gyakorlat, hogy a kezdő- és/vagy végdátumnak egy minden előfordulható dátumnál kisebb, vagy nagyobb dátumot választanak, és alapértelmezetten ezt az értéket veszi fel az ,,ismeretlen'' érték.
	A \tabla{nyilvántartás} táblában a \textit{mettől} érték a műszer dolgozónak történő átadás időpontja, de a \textit{meddig} értéke nem lehet egy ,,kellően távoli dátum, például: 9999.12.31'' mert ez nem a valós tényeket tükrözné. Ehelyett a \textit{null} érték megadása a megfelelő, hiszen a műszer felvételének pillanatában még tudjuk pontosan, mikor vesszük vissza azt.
	Az időfüggő ismeretek, mint például valami árának a kezelésére számos technika ismert. Az egyik ilyen technika, hogy például a termék aktuális árának rögzítése a termék táblában és egy másik táblában az árak változását kezeljük -tól -ig formában. Ebben az esetben az aktuális árat közvetlenül a termék tulajdonságaként értjük el, míg korábbi árak esetén az árak változását tartalmazó táblából kell kikeresni.
	}
}
\end{minipage}

\subsection{KALIBRÁLÁS tábla}
A \tabla{kalibrálás} táblában rögzítjük a műszerek kalibrálási adatait, annak eredményét. A kalibrálás során három kötelező és további három opcionális ellenőrző mérést kell vagy lehet végezni. Az ellenőrző mérések elvárt értékét a kalibráló határozza meg figyelembe véve az eszköz használati módját, és az elvért értékhez hasonlítja a mért értéket. A minősítést a műszer állapotától és az ellenőrző mérések eredményétől függően a kalibráló állapítja meg.
\\

\tabla{kalibrálás}(\pk{sorszám}, \fk{egyedi szám}, \fk{felhasználónév}, időpont, hőmérséklet, páratartalom, elvárt1, mért1, elvárt2, mért2, elvárt3, mért3, elvárt4, mért4, elvárt5, mért5, elvárt6, mért6, \fk{minősítés})\\



\subsection{FELJEGYZÉS tábla}

\subsection{FELJEGYZÉS tábla}

\subsection{NAPLÓ tábla}
