% Fontosabb fogalmak

\subsection{A modellben használt fontosabb fogalmak}

\paragraph*{törzsszám} A dolgozók egyedi azonosító jele a Humánügyi rendszerben. 

\paragraph*{titulus} A dolgozó neve előtt álló például \textit{dr.}, \textit{özv.}, \textit{ifj.} stb.
jelző.

\paragraph*{költséghely kód} A SAP rendszerből átvett, a költséghelyek azonosítására 
szolgáló kötött formátumú kód. Formája \textit{dd-ddd-d}, ahol a \textit{d} $0-9$ közötti
számot jelent.

\paragraph*{szerep} A rendszer felhasználóinak a rendszer szempontjából fontos
szerepköre. Az egyese szerephez más-más felhasználói funkciók tartozhatnak.

\paragraph*{aktív szerep} Egyes szerepek idővel megszűnhetnek, vagy átmenetileg
nincs szükség rájuk. Ez törlés helyett az \textit{aktív szerep} tulajdonságon keresztül vezérelhető.

\paragraph*{mérőeszköz minősítése} Az ellenőrzés, vagy kalibrálás során az adott
eszköz használhatóság megállapításának eredménye.

\paragraph*{kalibrálás} Olyan tevékenységek összessége, amellyel meghatározzuk az
összefüggést a mérőeszköz értékmutatása, valamint a mérendő mennyiség valós
tulajdonságai között. A kalibrálás során általában egy rendkívül pontos, úgynevezett
etalon mérőműszer értékeivel vetjük össze a vizsgált eszközt.

\paragraph*{felhasználónév} A rendszer használatához szükséges, a felhasználót a 
felhasználó jelszavával együtt azonosító jelsorozat.

\paragraph*{jelszó} A felhasználó által megadott jelsorozat. A jelszó egyirányú kódolás
után nyert képe kerül tárolásra.

\paragraph*{törzsadat} Törzsadatnak nevezzük azokat az adatokat, melyek leírják a közös
tulajdonsággal bíró egyedeket.

\paragraph*{cikkszám} A SAP rendszerben is használható cikket (terméket) azonosító
kód. Hossza maximum 12 karakter lehet, tartalmazhat betűt, számot, írásjelet, egyéb 
karaktereket. 

\paragraph*{mérőeszköz fajta} A mérőeszköz működési-kijelzési módjára utaló ismeret
(nóniuszos, mérőórás, \dots).

\paragraph*{mérőeszköz típusa} A mérés végrehajtásához szükséges alak, vagy forma meghatározása (csőrös, mélységmérős, \dots).

\paragraph*{osztás} A mérőeszköz fő skálájának legkisebb osztásának mértékegysége 
(mm, kg, univerzális, \dots)

\paragraph*{pontosság} A mérőeszközzel végzett mérés pontossága (0,01; 0,5; 10)

\paragraph*{mérési tartomány} Az eszközzel végezhető mérés alsó és felső határértéke.
(0-150, 50-75, 0-1500)

\paragraph*{mérési tartomány mértékegysége} A mérési tartományhoz rendelt
mértékegység. Nem feltétlenül azonos az osztás mértékegységével.

\paragraph*{kalibrálási időtartama} Azon napok száma mely két kalibrálás között eltelhet.

\paragraph*{gravírszám} A mérőeszközre gravírozott, vagy más maradandó módon felvitt
egyedi azonosító szám.

\paragraph*{gyári szám} A eszköz gyártó által meghatározott száma.

\paragraph*{leltári szám} A SAP rendszerben rögzített tárgyi eszköz azonosító szám.

\paragraph*{törzs márőeszköz} Azon eszköz, amellyel a többi mérőeszköz ellenőrzését 
végzik.

\paragraph*{üzembe helyezés dátuma} A mérőeszköz első használatra történő 
kiadásának dátuma, tárgyi eszköz esetén az aktiválás dátuma.

\paragraph*{selejtezésre felajánlás dátuma} az a dátum, amikor az eszközről kiállításra
kerül a selejtezési javaslat.

\paragraph{selejtezés dátuma} a Selejtezési Bizottság ,,Selejtezési Jegyzőkönyv'' 
dátuma.

\paragraph*{mérőeszköz státusza} a kalibráltság állapotát mutatja.

\paragraph*{belső kalibrálás sorszáma} az elvégzett kalibrálás sorszáma, később ebből képződik a kalibrálási jegyzőkönyv sorszáma. A kalibrálás során minimum három mérést
kell végezni, ha a vizuális vizsgálat alapján az eszköz nem azonnal selejt (például törött).

\paragraph*{elvárt érték} A kalibrálás során előre rögzített érték, méret, stb,, amelyet a kalibrálandó eszközzel megmérünk, ellenőrzünk.

\paragraph*{mért érték} a kalibrálás során mért, megfigyelt érték.

\paragraph*{külső kalibrálás sorszáma}  külső kalibrálók kalibrálási jegyzőkönyvének
jele, száma.

\paragraph*{partnerkód} külső kalibrálást végzők SAP-n belüli kódja.
mérőeszköz felügyelő:	a műszereket fogadja a kalibráláshoz, használat, állapot 
ellenőrzése, üzemszerű használat körülményeinek ellenőrzése, előkészítés selejtezésre.

\paragraph*{laborvezető}	felsőfokú végzettséggel(!) rendelkező személy, feladata a
metrológus és mérőeszköz felügyelő munkájának ellenőrzése, jóváhagyása. Selejtezés
jóváhagyása, kalibrálás ellenőrzése és esetenként kalibrál is.

\paragraph*{metrológus} feladata a kalibrálások elvégzése, bejövő eszközök
minőségének ellenőrzése, nyilvántartásba vétel (gravírszám kiosztása és gravírozása),
jelzi a raktárnak a gravírszámot (sarzs), (a raktár bevételezi), ha nem kerül felhasználásra, 
beviszi a raktárba és a helyére teszi (minden eszköznek a raktárban tárhelye van
tárhely:műszer $\rightarrow$ N:M), kalibrálási jegyzőköny készítése, jóváhagyatja, matricázza az eszközt. Selejt, vagy csere esetén a névhez hozzárendeli az új eszközt, a régit a nyilvántartásból leveszi a dolgozótól, raktárat értesíti.

\paragraph*{érvényesítő matrica} -- vagy kalibrálási matrica. Olyan fém fólia matrica, 
melynek felső részén az év, alatta négy részre osztott piros-zöld színű táblázat mutatja a
kalibrálás érvényességének negyedévét. A zöld színű negyedévek jelzik a kalibrálás
érvényességi idejét. Az érvényesség ideje a matrica alapján a tárgy negyedév utolsó
napja. 

\paragraph*{lekérdező}	\textit{csak} a rendszer adatinak lekérdezésére jogosult
felhasználó (például raktáros, tárgyi eszköz nyilvántartó, belső ellenőr, technológus\dots)

\paragraph*{javasolt ciklusidő} külső kalibrálási jegyzőkönyv tartalmazhat javaslatot a 
két kalibrálás között eltelt időre. 

\paragraph*{pontossági osztály} a mérőeszközök olyan csoportja, amely a hibák
specifikált határok között tartása céljából meghatározott metrológiai követelményeknek
eleget tesz. A pontossági osztályt rendszerint megállapodás szerinti számmal, vagy jellel
jelölik, és ezt osztályjelnek nevezik.