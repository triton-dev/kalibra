% Fontosabb fogalmak
Az adatbázistervezés egyik fontos eleme a fogalmi modell megalkotása. A fogalmi modell
a valóság kompromisszumoktól mentes képe\footnote{Legalábbis az adott probléma
megoldása tekintetében.}. A fogalmi modell készítése során a használt fogalmakat meg
kell magyarázni annak érdekében, mindenki számára egyértelmű legyen azok jelentése.

Ugyancsak a fogalmi modell jellemzője, hogy a modellben természetes és teljes
megjelöléseket használunk. Rövidíteni csak teljesen egyértelmű esetekben szabad.
Például fogalmi modellben a \textit{Költséghely kód} helyett még nem írható \textit{ktgkod}
rövidítés. 

\subsection{A modellben használt fontosabb fogalmak}
\paragraph*{törzsszám} A dolgozók egyedi azonosító jele a Humánügyi rendszerben. 
\paragraph*{titulus} A dolgozó neve előtt álló például \textit{dr.}, \textit{özv.}, \textit{ifj.} stb.
jelző.
\paragraph*{költséghely kód} A SAP rendszerből átvett, a költséghelyek azonosítására 
szolgáló kötött formátumú kód. Formája \textit{dd-ddd-d}, ahol a \textit{d} $0-9$ közötti
számot jelent.
\paragraph*{szerep} A rendszer felhasználóinak a rendszer szempontjából fontos
szerepköre. Az egyese szerephez más-más felhasználói funkciók tartozhatnak.
\paragraph*{aktív szerep} Egyes szerepek idővel megszűnhetnek, vagy átmenetileg
nincs szükség rájuk. Ez törlés helyett az \textit{aktív szerep} tulajdonságon keresztül vezérelhető.
\paragraph*{mérőeszköz minősítése} Az ellenőrzés, vagy kalibrálás során az adott
eszköz használhatóság megállapításának eredménye.
\paragraph*{kalibrálás} Olyan tevékenységek összessége, amellyel meghatározzuk az
összefüggést a mérőeszköz értékmutatása, valamint a mérendő mennyiség valós
tulajdonságai között. A kalibrálás során általában egy rendkívül pontos, úgynevezett
etalon mérőműszer értékeivel vetjük össze a vizsgált eszközt.
\paragraph*{felhasználónév} A rendszer használatához szükséges, a felhasználót a 
felhasználó jelszavával együtt azonosító jelsorozat.
\paragraph*{jelszó} A felhasználó által megadott jelsorozat. A jelszó egyirányú kódolás
után nyert képe kerül tárolásra.
\paragraph*{törzsadat} Törzsadatnak nevezzük azokat az adatokat, melyek leírják a közös
tulajdonsággal bíró egyedeket.
\paragraph*{cikkszám} A SAP rendszerben is használható cikket (terméket) azonosító
kód. Hossza maximum 12 karakter lehet, tartalmazhat betűt, számot, írásjelet, egyéb 
karaktereket. 
\paragraph*{mérőeszköz fajta} A mérőeszköz működési-kijelzési módjára utaló ismeret
(nóniuszos, mérőórás, \dots).
\paragraph*{mérőeszköz típusa} A mérés végrehajtásához szükséges alak, vagy forma meghatározása (csőrös, mélységmérős, \dots).
\paragraph*{osztás} A mérőeszköz fő skálájának legkisebb osztásának mértékegysége 
(mm, kg, univerzális, \dots)
\paragraph*{pontosság} A mérőeszközzel végzett mérés pontossága (0,01; 0,5; 10)
\paragraph*{mérési tartomány} Az eszközzel végezhető mérés alsó és felső határértéke.
(0-150, 50-75, 0-1500)
\paragraph*{mérési tartomány mértékegysége} A mérési tartományhoz rendelt
mértékegység. Nem feltétlenül azonos az osztás mértékegységével.
\paragraph*{kalibrálási időtartama} Azon napok száma mely két kalibrálás között eltelhet.
\paragraph*{gravírszám} A mérőeszközre gravírozott, vagy más maradandó módon felvitt
egyedi azonosító szám.
\paragraph*{gyári szám} A eszköz gyártó által meghatározott száma.
\paragraph*{leltári szám} A SAP rendszerben rögzített tárgyi eszköz azonosító szám.
\paragraph*{törzs márőeszköz} Azon eszköz, amellyel a többi mérőeszköz ellenőrzését 
végzik.
\paragraph*{üzembe helyezés dátuma} A mérőeszköz első használatra történő 
kiadásának dátuma, tárgyi eszköz esetén az aktiválás dátuma.
\paragraph*{selejtezésre felajánlás dátuma} Az a dátum, amikor az eszközről kiállításra
kerül a selejtezési javaslat.
\paragraph{selejtezés dátuma} A Selejtezési Bizottság ,,Selejtezési Jegyzőkönyv'' 
dátuma.
\paragraph*{mérőeszköz státusza} A kalibráltság állapotát mutatja.
\paragraph*{belső kalibrálás sorszáma} Az elvégzett kalibrálás sorszáma, később ebből képződik a kalibrálási jegyzőkönyv sorszáma.
\paragraph*{külső kalibrálás sorszáma} A külső kalibrálók kalibrálási jegyzőkönyvének
jele, száma.
\paragraph*{partnerkód} A külső kalibrálást végzők SAP-n belüli kódja.

