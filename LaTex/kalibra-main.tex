\documentclass[a4paper,12pt]{report}
\usepackage{amsfonts}
\usepackage{amssymb}
\usepackage[T1]{fontenc}
\usepackage[utf8]{inputenc}
\PassOptionsToPackage{defaults=hu-min}{magyar.ldf}
\usepackage[magyar]{babel}
\setcounter{secnumdepth}{5}
\usepackage{graphicx}

%opening
\title{%
    \Huge \textbf{Mérőeszköz kalibrálás \\ és nyilvántartó rendszer} \\
    \vspace*{72pt}
    \textbf{Rendszerterv}
    }

\author{Szikora György}
\date{2020}

\begin{document}

\maketitle

\tableofcontents

\part{Bevezetés}
\chapter{A tervezett rendszer}
A BKV Vasúti Járműjavító Kft. 9001-es ISO rendszerben végzi tevékenységeinek 
jelentős részét. A minőségbiztosítás fontos eleme a mérő-, ellenőrző- és 
vizsgáló eszközök, berendezések nyilvántartása, megfelelő működésük rendszeres 
ellenőrzése, a napi karbantartáson túlmutató ápolása, esetleges javítása, a mért 
értékek helyességének biztosítása, amit a kalibrálás, egyes esetekben a 
hitelesítés biztosít.

A Kft. az 90-es évek végétől a mérő-, ellenőrző-, viszgáló eszközök 
nyilvántartására egy mára elavultnak tekinthető programot használ, ami nem,
vagy alig biztosítja az elégséges feltételeket az idő közben életbe lépet 
24/2016 (VII. 18.) NFM rendelet követelményeinek betartására.

Az új rendszer tervezésekor figyelembe kell venni:
\begin{itemize}
\item a régi rendszerből átvehető adatokat, azok helyességét,
\item a régi rendszer használható funkciónak bevezethetőségét,
\item a felhasználói igényeket,
\item a vonatkozó, jelenleg hatályos jogszabályi és más követelményeket,
\item az új rendszer ne kötődjön egyetlen számítógéphez,
\item a rendszer adatai legyen kellően védettek,
\item az új rendszer üzembe állításához szükséges infrastrukturális és szoftver 
környezetet.
\end{itemize}

\chapter{A jelenlegi rendszer}
\section{A jelenlegi rendszer bemutatása}
A jelenleg használt mérőeszköz és azok kalibrálási állapotát tartalmazó rendszer 
egy mára már elavult Clipper, vagy dBase alkalmazás. Erre utalnak a Kft.
hálózati meghajtóján tárolt fájlok \textit{DBF}, illetve \textit{NTX} 
kiterjesztése.

A hálózatos kialakítás előnye, hogy elvileg minden olyan eszközről elérhető az
adatbázis, amely a keretprogramon keresztül képes csatlakozni hozzá. Egyúttal 
kockázatot jelent, hogy a hálózati meghajtón lévő fájlok egy alacsony szintű 
hozzáférés esetén is könnyen megváltoztathatók, tekintettel arra, hogy a dBase 
adatfájlok strukturált szerkezete szövegszerkesztő, táblázatkezelő 
alkalmazással egyszerűen módosítható.

A DBF fájlok tanulmányozása az alábbiakat tárta fel:
\begin{itemize}
\item az első bejegyzések 1999-ben keletkeztek,
\item az adatbázis inkonzisztenciájára utaló jelek láthatók,
\item egyes mezők kitöltése esetleges, nem következetes,
\item az adatok egy része meglévő Kft.-s nyilvántartásokra épül, például a
dolgozók azonosítása,
\item másrészt nem „SAP kompatibilis”, a SAP 2001-ben került bevezetésre, emiatt 
az egyik rendszerből kinyert adat direkt módon nem használható fel a másik 
rendszerben.
\end{itemize}
A keretprogram tanulmányozása rámutatott, hogy:
\begin{itemize}
\item a rendszerből releváns riportok kinyerésére csak képernyőkép útján van
lehetőség,
\item nincs mód az aktív-inaktív állapotok rögzítésére,
\item az adattáblák kulcs és leíró mezői keverednek.
\item az egyik fő hiba, hogy a jelenlegi ,,adatbázis'' inkonzisztens.
\end{itemize}

\section{A jelenlegi rendszer további használatának lehetősége}

A jelenlegi renszder javítására nincs mód, az eredeti forráskód nincs meg, csak
a futtatható állománnyal rendelkezik a Kft. 
A karaktergrafikus alkalmazás egyébként a feladat elvégzéshez tökéletes lenne, 
azonban a fejlesztői környezetet meg kell vásárolni, illetve licence díjat kell 
fizetni érte.
El kellene végezni az adatbázis újratervezést, mert mint a későbbiekben 
bemutatásra kerül, az adatbázis tervezésekor olyan hibákat követtek el az 
egykori tervezők, amelyek minden képpen javításra szorulnak.

A jelenlegi rendszer a működés folyamatosságának érdekében tovább használható, 
de egyidejűleg el kell kezdeni a kialakítandó rendszer tervezését.

A további inkonzisztencia növekedésének lassítása érdekében:
\begin{itemize}
\item lehetőleg kerülni kell új mérőeszköz típusok rögzítését,
\item  szöveges mezők kitöltésénél törekedni kell az azonos értékek azonos módon
történő rögzítésére,
\item az azonosító szerepű mezők kitöltése előtt ellenőrizni kell, hogy az 
azonosító létezik-e
\end{itemize}

A jelenlegi rendszer adatai részben felhasználhatók az új rendszerben is, 
illetve a jelenlegi sémát kiindulásként fel lehet használni az új rendszer 
tervezéséhez.

\section{A régi rendszer hibái, az újratervezés szükségessége}

\subsection{Kulcsok hiánya}
A relációs adatbázisok az \textit{egyed - tulajdonság - kapcsolat} hármasra
épülnek. \textit{Egyed}nek nevezzük az adatokat tartalmazó táblákat, melyek 
sorai az \textit{egyed} előfordulásokat tartalmazzák, a táblák oszlopai a 
\textit{tulajdonságok}, amelyek az egyed jellemzőit tárolják. A táblák 
közötti összefüggéseket, viszonyokat a \textit{kapcsolat} írja le.

Azokat a tulajdonságot, vagy tulajdonságokat, amelyek egyértelműen azonosítanak
egy-egy egyedelőfordulást \textit{kulcs}nak nevezzük. Ha \textit{kulcs} egy 
elemű akkor egyszerű, ha több elemű akkor \textit{összetett kulcs}nak nevezzük.

A táblák közötti kapcsolatokat a \textit{kulcs}ok biztosítják. Ehhez az egyik 
tábla kulcsát egy másik tábla tartalmazza és \textit{idegen-, vagy külső 
kulcs}nak nevezzük. Emellett kell állítani a szabályt aminek teljesülését az 
\textit{adatbázis motor} minden esetben automatikusan ellenőriz. Amennyiben a 
felállított szabály sérülne, úgy a motor nem engedi végrehajtani a kért 
műveletet.

A jelenlegi rendszerben, ha vannak is kulcsok, valószínűleg a szabályok leírása 
elmaradt,és semmi nem akadályozza meg, hogy a helytelen érték kerüljön a 
rendszerbe. Például a kalibrálást végezte oszlopban az érték hol 
\texttt{HARANGOZO}, hol \texttt{HRANGOZO}, de van \texttt{HARANGÓZO} és 
\texttt{HARANGÓZÓ} is.

\subsection{Kulcsok és szabályok hiánya, annak következményei}
A relációs adatbázisok táblái között a kulcsok írják le a kapcsolatot. Azzal, 
hogy az egyik tábla elsődleges kulcsát egy másik tábla idegen kulcsként 
tartalmazza, valamint a kulcsok között felállításra került a szabály, a táblák 
között kialkult a kapcsolat. Megfordítva, két tábla akkor van egymással 
kapcsolatban, ha az egyik tábla tartalmazza a másik tábla kulcsát.

Az összekapcsolás hiánya oda vezethet, hogy
\begin{itemize}
\item a felhasználó programnak kell ellenőrznie a kulcs szabályait,
\item ugyan ennek a programnak a feladata a szabály sérülésének kezelése,
\item ha a fentiek nem teljesünek, az adatbázis inkonzisztensé válhat.
\end{itemize}

Az inkonzisztencia veszélye abban nyilvánul meg, hogy lesznek olyan sorok az 
egyik táblában, amelyhez nem tartozik sor egy másik, a táblával kapcsolatban 
álló táblában.

A \ref{dolg} és \ref{kalib} táblák között a \textbf{dolgozókód} mező teremt
kapcsolatot. A DOLGOZÓ táblában ez a mező \textbf{kulcs}, míg a KALIBRÁLÁS 
táblában \textbf{idegen kulcs} szerepet tölt be.

\begin{figure}[ht!] \centering
\begin{tabular}{|l|l|}
        \hline
        \multicolumn{2}{|c|}{\textbf{DOLGOZÓ}}\\
        \hline
        \textbf{dolgozókód}&\textbf{dolgozónév}\\
        \hline
        HARANGOZO&Harangozó István\\
        \hline
        KOVACS&Kovács József\\
        \hline
        \end{tabular}
        \caption{DOLGOZÓ tábla}\label{dolg}
\end{figure}

\begin{figure}[ht!]\centering

\begin{tabular}{|r|c|l|}
        \hline
        \multicolumn{3}{|c|}{\textbf{KALIBRÁLÁS}}\\
        \hline
        \textbf{sorszám}&\textbf{...}&\textbf{dolgozókód}\\
        \hline
        1&\dots&HARANGOZO\\
        \hline
        2&\dots&HARANGOZO\\
        \hline
        \dots&\dots&\dots\\
        \hline
        104&\dots&KOVACS\\
\end{tabular}
\caption{KALIBRÁLÁS tábla}\label{kalib}
\end{figure}
        

A kulcsok közötti szabályok betartását a motor ellenőrzi, így a KALIBRÁLÁS tábla
\textbf{dolgozókód} mezője csak olyan értéket vehet fel, ami szerepel a DOLGOZÓ
tábla \textbf{dolgozókód} mezőjében. Vissza irányban is igaz, ha egy dolgozót 
törlünk a DOLGOZÓ táblából, nem maradhat olyan rekord a KALIBRÁLÁS táblában, 
ami a törölt dolgozóhoz tartozott. Megadható olyan szabály, ami törli az árván 
maradó sorokat, vagy a \textbf{dolgozókód} mező értékét \textit{null} értékre 
állítja, de olyan szabály is létezik, hogy nem törölhető az adat.
Ugyan ez igaz arra az esetre is, ha a dolgzó kódja megváltozik. Ilyen esetben 
a motor automatikusan kicseréli a KALIBRÁLÁS tábla \textbf{dolgozókód} mezőjét 
az új értékre.

Az adatok, leírók egyik különleges értéke a \textit{null} érték. Nem keverendő
össze a nulla $(0)$ szám értékkel, sem az üres értékkel, például karakteres
adat esetén a '' azaz az üres karakterrel. A \textit{null} érték azt jelenti, 
hogy az adott pillanatban a mező értéke még ismeretlen, vagy nem értelmezehtő.
A \textit{null} értékek kezelésére az adatbázis motorok nyújtanak 
szolgáltatásokat. De vannak egyéb megkötések, mint például a kulcs, vagy 
összetett kulcs eleme nem lehet \textit{null} értékű.

\subsection{Kulcsok (azonosítók) képzése}

\subsubsection{Belső képzésű azonosítók}
A belső képzésű azonosítükat az adatbázis motor állítja elő, mint például a 
belső sorszám jellegű, vagy véletlen szerű azonosítók. Ilyen esetekben a motor 
biztosítja, hogy egy táblán belül ne ismétlődjenek a kiosztott azonosítók még 
abban az esetben sem, ha az korábban törölve lett. Ehhez a motor 
rendszertáblákat használ, melyekhez a hozzáférést a rendszer korlátozza. 
Legtöbbször megadhatók viszont a kezdő- és végértékek esetleg a lépésközök, 
lekérdezhető az utolsó, vagy a következő azonosító.

\subsubsection{Képzett azonosítók}
Természetesen lehetőség van tetszőleges tartalamú azonosítók előállítására, a 
motor ilyen esetben is képes ellenőrzni azok egyediségét. A képzett azonosítók 
valamilyen szabály, vagy logika mentén kerülenk előállításra. Ilyen képzett 
azonosító a Kft. cikktörzsében a cikkszám, de ilyenek az EAN vonalkódrendszerek 
is.

Képzett azonosítók esetén könnyen abba a hibába eshetünk, hogy valamilyen leíró
tulajdonságokat az azonosítóba kódolunk. Tételezzük fel, hogy a cikkszám 7.
karaktere egy szám, értékkészlete 0-9 közötti értéket vehet fel és valamilyen 
jellemzőt ír le. Könnyen belátható, hogy 10 jellemzőt lehet ezzel a mezővel 
leírni, de a 11. jellemzőt a szabály megszegése nélkül már nem, nem elég hozzá a 
mező értékkészlete.

Másik előforduló jelenség, hogy a termékstruktúra alapján határozzák meg a 
cikkszámokat. Ezzel önmagában nincs probléma, ha kellő figyelmet fordítanak 
arra, hogy az általános, több helyen is elődorduló, vagy felhasználható 
kereskedelmi termékeket nem sorolják ide. Ezzel elkerülhető, hogy ugyan annak a 
terméknek több cikkszáma is legyen.

\subsection{A meglévő rendszerek azonosítói}

Minden alrendszer esetén sérül \textbf{az adatbázis - egy} elv. Azaz nem lenne 
szabad olyan rendszereket létrehozni, amelyek új adatbázisok létrehozásával 
járnak. Számos esetben mégis erre kényszerülünk, mert a meglévő rendszerek 
bővítése jelentős költséggel járna, például a SAP esetében, vagy a kezelendő 
adatok nem illeszkednek abba a rendszerbe, amibe kis ráfordítással de kezelni 
lehetne azokat. Pédául a humánügyi adatok kezelését biztosító rendszerbe 
nehezen képzelhető el egy mérőeszköz nyilvántartó rendszer integrálása.

Ilyen esetekben olyan közös azonosítókat kell választani, amivel biztosítható,
hogy az egyik rendszer kimenete a másik rendszer bemenete lehessen, azaz a 
\textit{master} rendszer kulcsának meg kell jelennie a \textit{slave} 
rendszerben is, méghozzás ugyan olyan módon, ahogy a \textit{master} 
rendszerben létezik.

\subsubsection{A humánügyi rendszer}
A humánügyi rendszerben minden dolgozó egyedi sorszámot kap. Ha egy dolgozó 
munkaviszony megszűnik, majd ezek után ismét felvételre kerül, új azonosító 
számot kap. Az azonosító neve a \textbf{törzsszám}.

\subsubsection{A SAP rendszer}

A SAP rendszerben a termékek azonosítására a \textbf{cikkszám} szolgál. Ezzel a
mezővel önmagában azonosíthatók az azonos termékek csoportja, de ha a sok 
azonos termék közül konkréten egy terméket (vagy termékcsoportot) kell 
azonosítani, legalább még egy azonosító szükséges. Ez a mező a \textbf{sarzs} 
mező. Ilyen esetben a \textbf{cikkszám+sarzs} együtt azonosítja a konkrét 
terméket. A SAP rendszerben mindkét mező tárolási hossza kötött, ezt az új 
rendszerben is figyelembe kell venni.


A költséghelyek azonosítására a \textbf{költséghely kód} mező szolgál. Ennek a 
mezőnek a SAP-n belül a tárolási hossza és a felépítése is kötött amit az új 
rendszerben majd figyelembe kell venni.

A SAP rendszerben a külső partnereket a szállítótörzsben, a vevőket a 
vevőtörzsben tarjuk nyilván. A megfelelő kulcsok a \textbf{szállítókód} és a 
\textbf{vevőkód}. Ha a szállító egyben vevő is, mind a két törzsben kap 
azonosítót.

\chapter{Az új rendszer tervezése}

Az új rendszer adatbázist használ, ezért először azt kell megtervezni. 
A tervezés során gondosan fel kell mérni az igényeket, meg kell fogalmazni a 
problémákat. Az adatok jellege és a közöttük lévő kapcsolatok meghatározása 
után következik az \textbf{adatmodell} létrehozása. 

Az adatmodell akkor tekinthető megfelelőnek, ha:
\begin{itemize}
 \item \textbf{átfogó}, azaz az adott problémára nézve minden lehetséges adatot
 és minden lehetséges kapcsolatot ábrázolni és kezelni képes,
 \item \textbf{valósághű}, azaz képes leírni az adott problémára nézve
 kompromisszumoktól mentesen a valóságot, valamint annak lényeges és tartós 
 összefüggéseit,
 \item \textbf{mentes a redundaciától}, normalizált, azaz minden adatot csak 
egyszer tárol,
 \item \textbf{következetes}, azaz a modell elkészítésekor azonos jelrendszert 
használ azonos dolgok ábrázolásához. Ezek lehetnek ábrák, szövegek, szabványos, 
vagy kvázi szabványos jelölések.
\end{itemize}
Ez az adatbázis fogalmi-logikai szintje. Nem tartalmazza a mezők típusát, 
hosszát, csupán leírja az egyedek (táblák) szerkezét és a táblák közötti 
kapcsolatokat.

A korszerű kezelők használata során a fizikai szinttel - az adatok tényleges, 
fájl szintű tárolásával - a tervezés során nem kell foglalkozni, az az 
adatbázis motor feladata.



\section{A probléma leírása}
A probléma leírását a műszerek kalibrlása és annak nyilvántartása szempontjából
vizsgáljuk. Az \textit{master} rendszerek folyamatai nem képzik a modell részét.
A Kft. nyilvántartásában több száz mérő-, ellenerző- és vizsgáló eszköz 
- műszer - található. Ezek egy részé a Kft. Szerszámraktárában tárolják, másik 
része a dolgozók számára használatra van kiadva, harmadrészt a kalibrálás, vagy 
hitelesítés elvégzése miatt a kalibrálónál, vagy külső félnél található. 
A műszerek kalibrálási, vagy hitelesítési idővel rendelkeznek. A lejárt idejű 
műszerekkel mérést végezni nem szabad. A lejárt kalibrálási, vagy hitelesítési 
idejű, vagy nem kalibrált, nem hitelesített műszereket az elős 
használat előtt kalibrálni, vagy hitelesíteni kell.A kalibrálásról, vagy 
hitelesítésről sorszámmal ellátott jegyzőkönyv készül. A jegyzőkönyv
az összehasonlító mérések eredményétől függöen három kategóriába sorolja a
műszereket:
\begin{itemize}
 \item kalibrált, vagy hitelesített,
 \item csak tájékoztató mérésre használható,
 \item selejt.
\end{itemize}
A megfelelt minősítéső műszerekre matrica kerül, míg a selejt műszerekről 
selejtezési javaslat készül. A selejt műszereket elkülönítetten kell tárolni, 
további használatra tilos azokat kiadni.
 
Évente egy alkalommal úgynevezett mérőeszköz rovancs keretében minden műszert a 
rovancsot végzők lista alapján megkeresnek, azonosítanak, a kopott, 
olvashatatlan érvényesítő matricákat pótolják. A listák készítése 
költséghely-dolgozó-műszer bontásban készül.
 
A Szerszámkiadóban a műszerek azonosított tárhelyeken vannak elhelyezve. 
Egy tárhelyen több műszert is lehet tárolni. A visszavett műszereket annak 
tárhelyére kell visszatenni. A Szerszámkiadóban tárolják az újonnan vásárolt, 
még ki nem adott műszereket, melyek kalibrálását az első kiadás előtt végzik 
el. A Szerszámkiadóban található műszerek, lehetnek lejárt kalibrálási idejűek 
is, kiadás előtt a kalibrálást el kell végezni.

Az egyes műszerek kalibrálási, vagy hitelesítési idejét jogszabályok, vagy egyéb
rendelkezések határozzák meg. Jogszabályi előírás esetén ettől eltérni nem lehet.
Az egyéb rendelkezések esetén figyelembe lehet venni a műszer használatának
gyakoriságát, a használat körülményeit az időtartam meghatározásánál. 
Így például az azonos paraméterekkel rendelkező műszer esetében az ,,A'' műszer
365 napos, míg a ,,B'' műszer csak 120 napos kalibrálási idővel is rendelkezhet.

A műszerekről készült kalibrálási adatokat a műszer selejtezését követően is 
meg kell őrizni, szükség szerint a kalibrálási jegyzőkönyvet változatlan 
adattartalommal is elő kell tudni állítani.

Egy-egy műszerhez több feljegyzés is tartozhat, amelyek valamilyen plusz 
információt tartalmaznak az adott műszerre vonatkozóan. 

Esetenként előfordul, hogy a Kft. külső felek számára végez kalibrlási 
feladatokat. A folyamat nem tér el saját tulajdonú műszerek kalibrálási 
folyamatától.

\section{Felhasználói szerepek}
A probléma leírásából kiolvasható, hogy többféle szerepkör is megjelenik. Így 
van
\begin{itemize}
 \item \textbf{adminisztrátor}, aki a rendszerben található törzsadatok 
kezelését végzi, listákat készít, gondoskodik a rendszer adatainak 
naprakészségéért,
\item \textbf{kalibráló}, aki a kalibrálásokt végzi, rögziti a kalibrálások 
eredményét, elvégzi az eszközök minősítését,
\item \textbf{lekérdező}, aki a rendszer adatait csak lekérdezi, semmilyen 
létrehozó, módosító, törlő funkcióval nem rendelkezik,
\item és kell lennie olyan nem nevesített szerepnek is, ami a rendszer belső 
feladatait látja el, mint például az adatbázis mentése, a naplózás elvégzése, 
stb\dots
\end{itemize}

\section{Az új rendszer táblái}
A probléma leírása alapján elkészíthető a táblák durva felsorolása elsődleges 
elnevezése, illetve a bennük tárolandó adatok vázlatos leírása. Ez a későbbiek 
során bővülhet, szűkülhet, a megnevezések változhatnak. A tervezésnek ebben a 
szakaszában még szabatos neveket használunk.
A táblák előzetes felsorolását \aref{tablak-0} táblázat tartalmazza.
A táblák közötti kapcsolatok \aref{tablak0-kapcs0} ábrán láthatók.
\begin{figure}[ht!]
\centering
\begin{tabular}{|c|l|}
\hline
\textbf{Tábla neve}&\textbf{Tartalma}\\
\hline
FELHASZNÁLÓ & a tervezett rendszer felhasználóinak adatai\\
\hline
SZEREP & felhasználói szerepek felsorolása\\
\hline
DOLGOZÓ & dolgozói törzsadatok\\
\hline
KÖLTSÉGHELY & költséghelyek felsorolása\\
\hline
CIKKTÖRZS & mérőeszközök általános tulajdonságai\\
\hline
MŰSZER & mérőeszközök egyedi tulajdonságai\\
\hline
NYILVÁNTARTÁS & melyik eszköz melyik dolgozónál volt, van\\
\hline
KALIBRÁLÁS & kalibrálási adatok\\
\hline
MINŐSÍTÉS & a minősítések felsorolása\\
\hline
FELJEGYZÉS & műszerekhez tartozó feljegyzések\\
\hline
NAPLÓ & a műveletek naplója\\
\hline
\end{tabular}
\caption{Az új rendszer táblái}\label{tablak-0}
\end{figure}
\\

\begin{figure}[ht!]\label{tablak0-kapcs0}
\centering
\includegraphics[width=13cm]{kepek/tablak0-kapcs0.png}
\caption{A táblák és a közöttük lévő előzetes kapcsolatok}
\end{figure}

A kapcsolatok ábrázolásánál most még foglalkozunk azzal hogy a kapcsolat 
kötelező-e, vagy opcionális, csupán jelezzük, hogy a táblák között van 
valamilyen kapcsolat. Nem jelezzük a kapcsolat fokát, ami lehet \textbf{1:1, 
1:n, m:n} sem.






%
%%
%%%
%%
% 
\end{document}
